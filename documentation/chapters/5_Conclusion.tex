\chapter{Conclusion}

The results of this experiment indicate that automating the data preprocessing phase using Qwen2.5 shows promise, particularly with simpler and smaller datasets. In the case of the \textit{Iris} dataset, which is relatively straightforward, the approach demonstrated the potential for generating Python code that performs several essential preprocessing tasks. However, when faced with more complex datasets, the results were less satisfactory. The LLM was not able to fully meet the preprocessing requirements in terms of accuracy and completeness, because of the challenges of handling larger, more varied datasets.

While the performance of Qwen2.5 in this experiment did not reach the desired level for all use cases, it is crucial to emphasize that the idea behind automating data preprocessing is highly feasible. The current limitations of the results should not overshadow the potential of this approach. In fact, the experiment has proven that, with improved focus and enhanced methodologies, it is possible to develop a system capable of automating data preprocessing with higher accuracy and efficiency.

One promising method to improve preprocessing automation would be to iterate the process multiple times on the same dataset. After each round of preprocessing, the model would be provided with new information from the partially preprocessed dataset, which could be used to refine subsequent preprocessing tasks. This iterative approach would allow the system to progressively improve its understanding of the dataset and make more informed decisions in later stages of preprocessing. This could potentially lead to a more robust and effective preprocessing pipeline, where each iteration builds on the insights gained from previous steps.

The future of this concept is particularly exciting when considering the growing capabilities of LLMs and other machine learning models. By fine-tuning these models specifically for data preprocessing tasks and leveraging more powerful computational resources, it is likely that significant improvements can be made. With specialized models designed for this purpose, the accuracy and efficiency of the automated preprocessing workflow can be greatly enhanced, making it a viable solution even for complex datasets.

From a business perspective, the automation of data preprocessing holds significant promise. In many industries, data is the cornerstone of decision-making, but the preprocessing phase is time-consuming, labor-intensive, and often prone to human error. Automating this phase would not only reduce the amount of time spent on repetitive tasks but also allow data scientists and analysts to focus their efforts on higher-level tasks, such as model development and strategic decision-making. By reducing the manual effort required in preprocessing, companies could benefit from increased productivity and more efficient use of their workforce.

Moreover, the time saved during preprocessing could lead to faster project turnaround times. This can be especially important in industries where timely insights are critical, such as finance, healthcare, and marketing. The automation of preprocessing could also lower operational costs, as fewer resources would be required for data cleaning and preparation. Additionally, with a more consistent and error-free data preparation process, the risk of producing misleading or inaccurate analysis could be minimized, leading to more reliable decision-making.

The ability to automate data preprocessing could also enhance the competitiveness of companies that adopt it. By streamlining data workflows and reducing the time between data collection and analysis, organizations could move more quickly in responding to market trends and customer needs. This agility is crucial in today's fast-paced business environment, where gaining a competitive edge often depends on the speed and accuracy with which data is processed and analyzed.

In conclusion, while the results of this experiment may not have reached the desired level of performance with more complex datasets, the potential for using Qwen2.5 to automate data preprocessing is enormous. With continued advancements in computational power, model specialization, fine-tuning and in using more advanced versions of Qwen2.5, this approach could become a key tool for businesses looking to optimize their data workflows. The advantages in terms of cost savings, increased efficiency, scalability, and competitive advantage make this a highly promising area for further exploration and investment. Automating data preprocessing could ultimately transform the way businesses handle their data, empowering them to make faster, more informed decisions while reducing operational overhead.

\section{Future Work}

% The results of this experiment provide a strong foundation for further exploration in the field of automating data preprocessing using large language models (LLMs). However, to achieve more robust and reliable outcomes, several key areas for future work should be prioritized.

% First, model fine-tuning stands out as a crucial next step. Qwen2.5, being a general-purpose language model, does not fully capture the domain-specific nuances of data preprocessing tasks. Fine-tuning it on domain-specific datasets, particularly those focused on data preprocessing, could enhance its ability to handle the complexities of larger and more varied datasets. A model trained with a more targeted set of preprocessing scenarios and solutions would likely demonstrate improved performance in more complex situations.

% Additionally, hyperparameter optimization could improve model performance significantly. Experimenting with different hyperparameters and model configurations can lead to better adaptation to specific applications and datasets, improving the overall accuracy and efficiency of preprocessing tasks.

% Another promising avenue is the development of an iterative preprocessing approach, where the model refines the dataset progressively over multiple rounds. After each preprocessing step, the model would be provided with new information from the partially preprocessed dataset, allowing it to adjust and improve its strategy in subsequent rounds. This iterative process could greatly enhance data quality, as each step would build upon insights from previous iterations, enabling the system to make more informed decisions and address data issues more comprehensively.

% Moreover, the integration of other models into the preprocessing pipeline could provide performance benefits. Using ensemble methods or combining the Qwen2.5.5 model with other state-of-the-art models may enhance preprocessing by leveraging the strengths of different approaches. This could result in better handling of complex tasks and improve overall system robustness.

% Addressing the limitations of computational resources is another important factor. As the model grows in complexity, leveraging more powerful hardware and optimized algorithms will be essential to improve the speed and efficiency of preprocessing tasks. As computational capabilities advance, the performance of LLMs in handling larger and more complex datasets will likely improve, making the approach more scalable and efficient.

% In addition, there is a need for research into automating feature engineering and improving the handling of missing, inconsistent, or erroneous data. 

% Furthermore, user feedback incorporation should be considered for future iterations. By developing mechanisms that allow for the inclusion of user feedback into the preprocessing pipeline, the system can continuously improve based on real-world needs and user experience. 

% Finally, the integration of this automated preprocessing system into existing data pipelines is a crucial next step. Embedding this system into business workflows will allow organizations to realize the full potential of automated data preparation, creating a seamless and efficient end-to-end process. As businesses look to scale their data operations, integrating such automation could help reduce the burden of manual data preparation, accelerating the entire data processing pipeline.

In summary, while the current experiment offers a promising proof of concept, future work should focus on improving the model's ability to handle complex datasets, incorporating iterative preprocessing techniques, fine-tuning the model for domain-specific tasks, and addressing computational and integration challenges. The addition of features like hyperparameter optimization, user feedback loops, and ensemble modeling approaches will further enhance performance. With continued advancements in AI-driven automation, the potential for fully automating the data preprocessing phase holds great promise, offering substantial business benefits in terms of efficiency, consistency, cost reduction, and scalability.

%\subsection{Future Work}

%While the results obtained are promising, they indicate potential for further improvement. Future work could focus on several key areas:

%\begin{itemize}
    %\item \textbf{Model Fine-tuning}: Enhancing the model's performance through additional fine-tuning on domain-specific datasets could yield better accuracy and efficiency in preprocessing tasks.
    
    %\item \textbf{Hyperparameter Optimization}: Experimenting with different hyperparameters and model configurations may improve performance metrics and adapt the model more effectively to specific applications.
    
    %\item \textbf{Integration with Other Models}: Exploring the use of ensemble methods or combining Qwen2.5.5 with other state-of-the-art models could enhance performance through the benefits of diverse modeling approaches.
    
    %\item \textbf{User Feedback Incorporation}: Developing a mechanism to incorporate user feedback into the preprocessing pipeline can lead to iterative improvements, making the application more responsive to real-world needs.
    
    %\item \textbf{Extended Features}: Adding more preprocessing functionalities, such as advanced text augmentation techniques and context-aware normalization, could significantly enhance data quality and usability.
%\end{itemize}

%Overall, this project demonstrates the potential of using advanced language models like Qwen2.5.5 in real-world applications. By automating critical phases such as data preprocessing, we can improve efficiency, maintain consistency, and ensure data privacy, which is fundamental for firms. The open-source nature of Qwen2.5.5 fosters collaboration and innovation within the community, paving the way for further advancements in natural language processing and machine learning.